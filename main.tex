\documentclass[dvipdfmx,uplatex,titlepage]{jsarticle}
\usepackage[dvipdfmx]{graphicx}
\usepackage{amsmath}

\title{言語処理系 レポート課題1}
\author{1w182332 矢田宙生}
\date{課題出題日:2021年4月30日,課題提出日:2021年5月1日} 
\begin{document}
\maketitle

\section{概要}
本実習では、Raspberry Pi4にインストールした「Ubuntu 20.04.2 LTS」上で、
\begin{itemize}
  \item 字句解析器の生成器であるflexを用いてC言語の字句解析プログラムを生成
  \item 構文解析器の生成器であるbisonを用いてC言語の構文解析プログラムを生成
  \item 動作確認
\end{itemize}

を行った

本レポートでは、動作確認の結果を示す。

\section{動作確認}

\$ flex tinycalc.l \#字句解析器の生成

\$ bison tinycalc.y \#構文解析器の生成

\$ gcc -o sample02 tinycalc.tab.c -ly -lfl \#コンパイル

\$ ./sample02 \#実行

1+1*8/2-3+3*2

8

\$ ./sample02 \#実行

1+1+8+2+3+3+2

20

計算結果が正しく出力されることを確認できた。
\end{document}